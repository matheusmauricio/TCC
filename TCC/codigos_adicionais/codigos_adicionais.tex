

% Codigo para citações longas (com mais de 3 linhas)
\begin{flushright}
\begin{minipage}{0.8\textwidth} % 75% de 160; margem para citações longas
\begin{quote}
\singlespacing % Código para colocar espaçamento simples em uma citação longa
"O importante é não parar de questionar. A curiosidade tem sua própria razão para existir. Uma pessoa não pode deixar de se sentir reverente ao contemplar os mistérios da eternidade, da vida, da maravilhosa estrutura da realidade. Basta que a pessoa tente apenas compreender um pouco mais desse mistério a cada dia.
Nunca perca uma sagrada curiosidade". Albert Einstein
\end{quote}
\end{minipage}
\end{flushright}


% ---------------------------------------------------------------------------------

% Código para colocar figuras. Lembrando que TODAS as figuras do trabalho devem estar salvas na pasta IMAGENS. 
\begin{figure}[!h]
\centering
\includegraphics[width=0.6\linewidth]{imagens/ciclolatex.pdf} % Aqui é o tamanho da figura, assim como onde está salva a figura. Mude conforme a sua necessidade.
\caption{Ciclo de vida de um arquivo Latex. \\\hspace{\textwidth}Fonte: Alberto Simões. Mini-Curso Latex. Disponível em $<$alfarrabio.di.uminho.pt/~albie/lshort/presentation.pdf$>$.} % Aqui é a legenda. Mude de acordo com a sua necessidade.
\label{ciclo}

\end{figure}