\begin{titlepage}
	
	
	\setcounter{page}{7} % Define qual � o n�mero da p�gina do cap�tulo
	\begin{center}
		
		\textbf{ABSTRACT} % Este comando � utilizado para criar cap�tulos
		
	\end{center}
	
	\onehalfspacing % Espa�amento de 1,5
	
	Whitin a city, there are many infrastructure, safety and health problems that affect the population.  Identifying, analyzing and solving these problems require a lot of time, expenses and attention from the responsible agency for carrying out this inspection. Such resources could be minimized with the help of citizens by a computerized system, in addition to the traditional communication channels.
	\\This work has as finality to create an application for mobile devices with Android operating system, which would be able to alert citizens, through photographs attached to a city map, of the problems identified by them. These photographs will be available until the moment the responsible agency resolves the problem and mark it as corrected in the application. %While it is not fixed, other users can mark the problem as \textquotedblleft exist\textquotedblright \space or \textquotedblleft does not exist\textquotedblright, confirming or not the authenticity of the problem.
	While it is not fixed, other users can contribute with the confirmation or not of the veracity of the problem.
	\\This participation of the users is considered a way of crowdsourcing, that is to say, through a joint action of the citizens it is possible to guarantee the existence of a problem in a certain map marking, without the necessity to have a specialist in that region.
	
	
	
	\noindent
	\textbf{Keywords}: Application, problems, city, crowdsourcing, ombudsman.
	
	
	\vspace{1cm} % Espa�amento vertical de 1 cent�metro
	
\end{titlepage}