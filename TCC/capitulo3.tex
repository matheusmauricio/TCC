% =======================  C�digo do Cap�tulo 1. ==================================
%Se precisar gerar outros cap�tulos use a op��o %Arquivo > Salvar Como e salve o novo arquivo com outro nome. Assim como este arquivo, %sugerimos seguir com a mesma linha de racioc�nio e salvar como capitulo2 e assim por diante. %N�O SE ESQUE�A DE COLOCAR NO ARQUIVO MODELOIFES.TEX O COMANDO %\INSERT{NOMEDOARQUIVODOCAP�TULO}. =============================



\begin{titlepage}


\setcounter{page}{13} % Define qual � o n�mero da primeira p�gina do cap�tulo

\chapter{\textbf{AS NORMAS DO IFES}} % Este comando � utilizado para criar cap�tulos
\onehalfspacing % Espa�amento de 1,5

\vspace{0.5cm} % Aqui � configurado os 2 espa�os de 1,5

As normas do Instituto Federal do Esp�rito Santo (IFES)\cite{Ifes}, no intuito de padronizar a
formata��o de trabalhos cient�ficos realizados na institui��o. As normas tem por objetivo auxiliar os servidores e o corpo discente no processo de elabora��o desses trabalhos. As normas do IFES s�o baseadas nas recomenda��es da Associa��o Brasileira de Normas T�cnicas (ABNT)\cite{ABNT} e, al�m disso, foram estabelecidas adapta��es de acordo com a realidade da Institui��o.

Abaixo s�o listados os diversos tipos de textos acad�micos e cient�ficos que s�o padronizados pelas normas do IFES:

\begin{itemize}
\item Monografias (Trabalho de conclus�o de curso - TCC, trabalho de conclus�o de curso de especializa��o e/ou aperfei�oamento
\item Disserta��es de Mestrado
\item Teses de Doutorado
\item Relat�rios
\item Projetos de Pesquisa
\item Trabalhos Curriculares
\end{itemize}

\end{titlepage}